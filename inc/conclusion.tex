\section*{ЗАКЛЮЧЕНИЕ}
\addcontentsline{toc}{section}{ЗАКЛЮЧЕНИЕ}
\addtocounter{section}{1}
\setcounter{subsection}{0}

В квалификационной работе предложена и реализована оптимизация времени доступа к полям Java
объекта на примере поддержки в Java новой структуры данных FlatArray. С точки зрения
программиста эта структура выглядит как обычный Java класс, но реализованный внутри Java машины.
В работе были решены следующие задачи:
\begin{enumerate}
	\item Создано и протестировано несколько версий FlatArray с различной раскладкой полей
	\item Реализована загрузка классов наследников FlatArray
	\item Устранены проблемы с плохой оптимизацией nullcheck и typecheck при доступе к элементу FlatArray
	\item Реализована поддержка сборки мусора для FlatArray
	\item Разработаны и измерены тесты производительности 
	\item FlatArray интегрирован в JDK.
\end{enumerate}

Результаты проведенного тестирования показали, что хотя предложенное решение и не является универсальными,
оно дает некоторый прирост производительности в определенных сценариях. Например, FlatArray
может стать заменой обычному Java массиву объектов в тех случаях, когда бутылочным горлышком
горячего кода является доступ к памяти.

В качестве направления дальнейших исследований по теме квалификационной работы можно назвать
ослабление ограничений на использование структуры FlatArray, а также изучение возможностей
применения ее для построения других сложных структур данных помимо HashMap.  

Выпускная квалификационная работа выполнена мной самостоятельно и с соблюдением правил профессиональной этики. Все использованные в работе материалы и заимствованные принципиальные положения (концепции) из опубликованной научной литературы и других источников имеют ссылки на них. Я несу ответственность за приведенные данные и сделанные выводы.

Я ознакомлен с программой государственной итоговой аттестации, согласно которой обнаружение плагиата, фальсификации данных и ложного цитирования является основанием для не допуска к защите выпускной квалификационной работы и выставления оценки «неудовлетворительно».

\begin{tabular}{lp{2em}l} 
	\hspace{5cm}   && \hspace{4cm} \\ \cline{1-1}\cline{3-3} 
	ФИО студента   && Подпись студента
\end{tabular}

\begin{tabular}{lp{2em}l} 
	\hspace{5cm}   && \hspace{4cm} \\ \cline{1-1}
	\size{10pt}{(заполняется от руки)}   && 
\end{tabular}

\clearpage
