\section{Заключение}
В результате данной работы была реализована структура данных под названием FlatArray. С точки зрения Java программиста,
она представляет собой обычный Java класс, однако он реализуется внутри JVM.В данной работе были решены следующие проблемы:
\begin{enumerate}
	\item Создано несколько версий FlatArray с различной раскладкой полей
	\item Реализована загрузка классов наследников FlatArray
	\item Устранены проблемы с плохой оптимизацией nullcheck и typecheck при доступе к элементу FlatArray
	\item Реализована поддержка сборки мусора для FlatArray
	\item Разработаны и измерены тесты производительности 
	\item FlatArray интегрирован в JDK.
\end{enumerate}
\par
Данная структура данных показала хорошую производительность на ограниченном наборе тестов. этих тестов показал, что FlatArray может стать заменой обычному Java массиву объектов в тех случаях, когда бутылочным горлышком горячего кода является доступ к памяти.
\par
Выпускная квалификационная работа выполнена мной самостоятельно и с соблюдением правил профессиональной этики. Все использованные в работе материалы и заимствованные принципиальные положения (концепции) из опубликованной научной литературы и других источников имеют ссылки на них. Я несу ответственность за приведенные данные и сделанные выводы.
\par
Я ознакомлен с программой государственной итоговой аттестации, согласно которой обнаружение плагиата, фальсификации данных и ложного цитирования является основанием для не допуска к защите выпускной квалификационной работы и выставления оценки «неудовлетворительно».

\begin{tabular}{lp{2em}l} 
	\hspace{5cm}   && \hspace{4cm} \\ \cline{1-1}\cline{3-3} 
	ФИО студента   && Подпись студента
\end{tabular}

\clearpage
