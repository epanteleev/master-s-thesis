Современные объектно-ориентированные языки программирования, такие как Java, предоставляют удобные механизмы для написания программ. Обычно приложения представляют собой огромное число различных объектов. 
\par
Однако такая декомпозиция программ может приводить к негативному влиянию на производительность. Огромное число мелких объектов приводит к фрагментации памяти, плохой локальности процессорного кеша, а также к увеличению числа чтений указателей из памяти. Эти факторы неизбежно приводят к ухудшению производительности Java приложения.
\par
В этой работе предлагается способ, как можно уменьшить негативное влияние вышеперечисленных недостатков введением в язык Java новой структуры данных, названной flattened array или FlatArray. Она представляет собой последовательность объектов, расположенных в одном линейном участке памяти. 
Для Java программиста flattened массив имеет интерфейс обычного Java класса.
\par
Flattened array может быть применен для реализации многих других структур данных и алгоритмов, зависящих от скорости доступа к памяти. В частности, в данной работе была реализована альтернативная версия хеш-таблицы, которая в некоторых случаях показала лучшую производительность операции поиска.
\clearpage

